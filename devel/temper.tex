
\documentclass[11pt]{article}

\let\code=\texttt

\begin{document}

\title{Adding Simulated Tempering}

\author{Charles J. Geyer}

\maketitle

We want to add simulated tempering (a.~k.~a.\ serial tempering) to
this package (CRAN package \code{potts}).
We do not particularly want to add parallel tempering.

Simulated tempering and parallel tempering are implemented in R function
\code{temper} in R package \code{mcmc} but the same authors as this package.
The two packages have different licenses, \code{potts} is GPL and \code{mcmc}
is MIT, but presumably the authors can steal their own code and re-license it.

The fundamental problem is that C function \code{potts} is not modular.
It is structured (in the sense of "structured programming") but there
are no function calls.  It is just one big mess of loops inside of loops
inside of loops.

In order to allow us to have two C functions, \code{potts} and \code{temper}
one that does not do tempering (like the current function) and one that does
but are otherwise identical, we need functions that do
\begin{itemize}
\item the basic Swendsen-Wang update and
\item the basic simulated tempering update (change of distribution).
\end{itemize}
And these can be wrapped with the loop inside of loop inside of loop that
does the MCMC with batching and spacing.

We also have the issue of all of the packing and unpacking of pixel data
(and bond data?).  Perhaps that should be modularized too.

Just say no!  Like what was done to add an output function to the code,
we will add simulated tempering in the kludgiest possible way.  We will
add arguments to C function \code{potts}


\end{document}

